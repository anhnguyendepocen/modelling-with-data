\begin{frontmatter}
	
	%% Title, authors and addresses
	
	%% use the tnoteref command within \title for footnotes;
	%% use the tnotetext command for the associated footnote;
	%% use the fnref command within \author or \address for footnotes;
	%% use the fntext command for the associated footnote;
	%% use the corref command within \author for corresponding author footnotes;
	%% use the cortext command for the associated footnote;
	%% use the ead command for the email address,
	%% and the form \ead[url] for the home page:
	%%
	%% \title{Title\tnoteref{label1}}
	%% \tnotetext[label1]{}
	%% \author{Name\corref{cor1}\fnref{label2}}
	%% \ead{email address}
	%% \ead[url]{home page}
	%% \fntext[label2]{}
	%% \cortext[cor1]{}
	%% \address{Address\fnref{label3}}
	%% \fntext[label3]{}
	
	\title{Mathematical epidemiology in a data-rich world}
	
	%% use optional labels to link authors explicitly to addresses:
	%% \author[label1,label2]{<author name>}
	%% \address[label1]{<address>}
	%% \address[label2]{<address>}
	
	\author{Julien Arino}
	
	\address{Department of Mathematics \& Data Science NEXUS, University of Manitoba, Winnipeg, Manitoba, Canada}
	
	\begin{abstract}
		%% Text of abstract
		I discuss the acquisition and use of ``background'' data in mathematical epidemiology models, advocating a pro-active approach to the incorporation of said data. I illustrate various mechanisms for acquiring data, mainly from open data sources. I also discuss incorporating this data into models.
	\end{abstract}
	
	\begin{keyword}
		Mathematical epidemiology \sep Data acquisition \sep Open data
		%% keywords here, in the form: keyword \sep keyword
		
		%% MSC codes here, in the form: \MSC code \sep code
		%% or \MSC[2008] code \sep code (2000 is the default)
		
	\end{keyword}
	
\end{frontmatter}
